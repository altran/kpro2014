\documentclass[../document.tex]{subfiles}
\begin{document}
\section*{Abstract}
In this project we were tasked to create ideas and make an implementation to visualize a concept known as “Internet of Things”. The “Internet of Things” is a concept in which all devices are interconnected in some way and most likely connected to the internet as well. In our specific project we will be using sensors. The sensors are small and cheap, and they are to be attached to various places in an environment. The sensors are supposed to communicate together in a network and communicate with a central hub. The data they measure is collected from the environment: temperature, lighting, humidity and more. We are supposed to visualize this data in an interesting and presentable way.

The main challenge of this project was collecting requirements, creating the architecture and thinking of ideas to represent the data in an interesting fashion. The project itself is very abstract and the field is fairly new to our customer, Altran AS. Consequently, our project is not an end product, but rather what a company would do internally to test a new product. This also provides some restrictions. We need to use a REST based architecture to communicate between the database and the central hub. Furthermore, because the customer wants to be able to showcase our implementation he needs to be able to access an always movable database, or one that is stored on the internet using cloud storage. This limits both the implementation and the use of our database.

To overcome these challenges we needed to brainstorm ideas as to how our implementation can be made around the requirements. As the data is quite static, making an interesting implementation is challenging. We thought of a few ideas and presented them to the customer. We listened to the responses the customer had about each idea, and decided to use a combination of ideas for our final implementation. We will use basic colours and shapes to visualize the data on a blank canvas.

We have also chosen to work with the Scrum model. Agile methodology is very efficient in this project; it is quite a conceptual project where the customer is not giving the development team a clear answer, but rather, we need to come up with an answer and improve it over time, or over the different sprints. While we do not follow our methodology fully when it comes to the documentation, the implementation will use the Scrum model fully. In addition, we will use pair programming to insure better quality and an equal spread of work experience between the group members.

Finally, we have learned that communication is key when dealing with customers that are unsure about what they want as an end goal. The project experienced changes as it was maturing and some of them were quite large, causing us to redo much of our work. In the future we will put communication as a top priority, especially when the field of work is new to the customer.

\end{document}