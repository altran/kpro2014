\documentclass[../document.tex]{subfiles}
\begin{document}
\section*{Abstract}
In this project we were tasked to create ideas and make an implementation to visualize a concept known as ``Internet of Things''. The ``Internet of Things'' is a concept in which multiple embedded devices are interconnected in some way and most likely connected to the internet as well. In our specific project we will be using sensors; small and cheap, they are to be attached to various places in an environment. The sensors are supposed to communicate together in a network and communicate with a central hub. The data they measure is collected from the environment: temperature, lighting, humidity and more. We are supposed to visualize this data in an interesting and presentable way.

The main challenge of this project was collecting requirements, creating the architecture and thinking of ideas to represent the data in an interesting fashion. The project itself is very abstract and the field is fairly new to our customer, Altran AS. As such, our project is not an end product, but rather what a company would do internally to test a new product. This also provides some restrictions; we need to use a REST based architecture to communicate between the database and the central hub. The database itself is also limited. As the customer wants to be able to showcase our implementation, the database should be accessible from any location. The database might be stored on the internet using cloud storage, which would limit the uses somewhat.

To overcome these challenges we needed to brainstorm ideas on an implementation that satisfied the requirements. The environment properties collected are quite static, making it challenging to develop an interesting implementation. After documenting the ideas, we presenting them to the customer to get feedback. We considered positive and negative sides of each idea, and decided to use a combination of ideas for our final implementation. The final implementation should use basic colours and shapes to visualize the data on a blank canvas.

We have also chosen to work with the agile methodologies of the Scrum model. This will likely be efficient, as our project task involves working with conceptual ideas, and is not strictly defined. We will need to come up with answers and improve our product over time, or over the different sprints. While we do not follow our methodology fully when it comes to creating documentation, the implementation phases will use the Scrum model fully. In addition, we will use pair programming to ensure better code quality, and to spread system knowledge equally amongst the group members.

Lastly, we have learned that communication is a key issue when given loose product specification, especially if the project is also exploring a field of work new to the customer. As our project was maturing, changes were made to the product requirements, some of them quite large, which caused the need to redo much of our work. In future projects we will put communication as a top priority, especially when the the field of work is new to the customer.

\end{document}
