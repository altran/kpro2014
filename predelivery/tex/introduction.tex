\documentclass[../document.tex]{subfiles}
\begin{document}
\section{Introduction}


\subsection{Project Name}

The project given by the customer is labeled ``Internet of Things''. ``Internet of Things'' is the conceptual idea of utilizing many embedded devices connected to each other and the internet. The goal of our project is to create an Internet of Things \emph{showroom}, that showcases the Internet of Things.


\subsection{Project Sponsor}

The customer for this project is Altran Norge AS. They are a part of an international group that offers innovative and high-tech engineering consulting. According to their website, their mission is to assist organizations and companies in creating and developing new products and services. They are well established in Enterprise Content Management, as well as embedded software and electronics design. They also offer other services in IT and technology. Altran Norge AS is responsible to provide the necessary hardware for the project.

\subsection{Project Purpose}
The goal of our project is to create an Internet of Things showroom. The showroom will illustrate the possibilities opened up by having many internet-connected embedded devices, and inspire new ideas. In the showroom, cheap sensors will be used to collect measurements of different environment properties, like the room temperature, humidity, or light level. This data is put together and presented in a visually appealing and interesting visualization. Ideally, the visualization will be interactive by responding to manipulation of the sensors. An ad-hoc wireless network will connect the sensors to a central hub, which will receive the collected data.

\subsection{Involved Parties}
Here we detail the parties involved in this project, in what way they are involved, their contact information, and their role within the project.

\subsection{The Customer}
\begin{tabular}{ll}
\hline
Name					&	Stig Lau\\
Company				&	Altran Norge AS\\
Position				&	Consultant\\
Contact Information		&	Stig.Lau@altran.no\\
\hline
\end{tabular}

\subsection{Students}
\begin{tabular}{ll}
\hline
Name				&	Contact Details\\ \hline
Besleaga, Catalin		&	catalinb@stud.ntnu.no\\
Ludvigsen, Niklas		&	niklasl@stud.ntnu.no\\
Kirkhus, Julie Johnsen	&	juliejk@stud.ntnu.no\\
Remøy, Maria Aune		&	mariaare@stud.ntnu.no\\
Sun, Shimin			&	shimins@stud.ntnu.no\\
Sutterud, Audun		&	audunsu@stud.ntnu.no\\
Zikic, Nenad			&	zikic@stud.ntnu.no\\
\hline
\end{tabular}

\subsection{Group Adviser}
\begin{tabular}{ll}
\hline
Name				&	Mohsen Anvaari\\
Position			&	Adviser, PhD-Student\\
Contact Information	&	mohsena@idi.ntnu.no\\
\hline
\end{tabular}

\subsection{Project Background}
The ``Internet of Things'' is a project that is in its early stages. The project is anticipated to change the world by implementing smarter systems. These systems are sensors that will detect different changes within an environment, and communicate this data to a central hub. This hub will analyze the data and allow us to manipulate the environment around us. A central issue in the project is the capabilities of the sensors. Which systems we can improve or replace and where we can create new opportunities. It is also necessary for us to be able to showcase this data visually and, depending on the system, show how the system works in real time.

The main focus of the project is the visualization of data gathered by the sensors. A screen will be used to display the visualization, and the visualization is directly influenced and manipulated in real-time by the data collected from the sensors. It is required that the visualization is interesting, that it provides opportunities for intuitive interaction, and that it directly showcases the ``Internet of Things'' concept by showing how the multiple sensors work together to gather information from the environment.

\subsection{Project Goals}
The overarching goal of this project is to develop ideas for visualizing the data gathered by the sensors in an interesting fashion. Another goal is develop a highly modular system infrastructure. This is to ensure that the system may be re-used in any sort of environment. It is also necessary to note where this system cannot be used, and where the current systems are already a better solution than our own system.

The specific goal of our team is to design a showcase, or visual representation of data. This data is gathered either in real-time or through the course of a day and shown in a time lapse. How the data is gathered is depending on what type of data we would like to showcase. 

\subsection{General Terms}
The main limitation of this project is the lack of hardware. The implementation and testing phases are dependent on the availability of sensors, but the sensors are currently in production and there is no guarantee as to when they are ready. Another limitation is time. In order to finish this project in a satisfying condition, we have to manage our working time well.

\subsection{Similar Projects}
As the project is fairly abstract, it is necessary to consider many variables and fields of study in order to fully investigate the project possibilities. There exists companies that are involved in the business of ``Internet of things'', and we will research those further in our documentation. However, because our goal is to present the data in a visual way, and not actually use the data in any other form, similar projects will often have little or few similarities to our project. This issue will be more concretely addressed in the part of the project dealing with preliminary studies.

\end{document}
