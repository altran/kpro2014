\documentclass[../document.tex]{subfiles}

\begin{document}

\section*{Choice-of-lifecycle Model}
\addcontentsline{toc}{section}{Choice-of-lifecycle Model}
Our choice of lifecycle is the Agile model, or specifically the Scrum model as it is outlined below. In the report, the choice of lifecycle model will be included with the preliminary study.

\subsection*{The Scrum model}
The Agile Manifesto and its twelve principles defines a new approach to software development that strongly contrasts with the previous Waterfall model. While the Waterfall model focuses on producing extensive documentation before implementation, an agile development process places the main focus on customer collaboration, presenting working software early, and responding to changes in the product requirements.

The Scrum model is a framework for software development that closely adheres to the Agile Manifesto. It is an incremental approach in that it develops the product through so-called sprints. A sprint is a phase of implementation that can last from 7 to 30 days. The input to each sprint is the product backlog, which is a list of all product requirements to be implemented ordered by priority. The sprint phase begins with a sprint meeting in which requirements are selected from the backlog to form a sprint backlog. The sprint backlog is the set of requirements that should be implemented during the sprint phase. During the sprint, the development team starts each day with a scrum meeting. Here, each team member presents what was done yesterday, what should be done today, and if there are any difficulties. The sprint ends with an end meeting in which the work progress is evaluated, and a demo of the new product increment is presented to the stakeholders.

In our initial project phase, we do not follow the Scrum model strictly. The initial phase is more concerned with documenting product requirements. This work would benefit from a close collaboration with the customer, making us choose our own agile model of work. In addition to having regular customer meetings, we opened up all our product documentation to the customer. The customer was prompted by email to review new versions of documents so that he could provide feedback outside meeting hours.

During the implementation phases, the Scrum model will be much more central. Here we will apply it to have a backlog, scrum meetings, and sprints.

\end{document}
