\documentclass[../document]{subfiles}

\begin{document} 

\subsection{List of Third Party Tools}
\label{third_party_tools}

In this appendix, we will list all the third party tools we have used throughout the length of this project, including their creators or homepages on the internet.

\paragraph{Google Drive} - As a part of a google account, \url{https://drive.google.com/} can be used to store documents and images in various formats. The most powerful ability of google drive is allowing multiple users to work on the same document at once.

\paragraph{Gmail} - A free email account with gigabytes of storage, \url{https://mail.google.com/}. It can be set up to forward all received email to multiple other gmail accounts, making it easy to send emails to a group of people.

\paragraph{Git} - Can be found here, \url{http://git-scm.com/} is a free, open-source, distributed version control system for both small and large projects.

\paragraph{GitHub} - Can be found here, \url{https://github.com/}. It is a free hosting service for public Git-repositories. Students can also get private repositories hosted free. It provides a web interface to browse repositories and branches.

\paragraph{TortoiseGit} - Can be found here, \url{https://code.google.com/p/tortoisegit/} is a free windows shell interface to Git.

\paragraph{SourceTree} - Can be found here, \url{http://www.sourcetreeapp.com/} is also a shell interface to git.

\paragraph{Tex Live} - Tex Live can be found here, \url{https://www.tug.org/texlive/}. It is an environment for creating LaTeX documents, and involves a free LaTeX compiler, and the Tex editor TeXworks. LaTeX documents are used in academics often, to make reports and papers look good and read well.

\paragraph{Skype} - Can be found here, \url{http://www.skype.com/} is a free internet calling program. Skype was required to hold meetings with the customer, who was situated in Oslo during the duration of the project.

\paragraph{yEd Graph Editor} - Can be found here, \url{http://www.yworks.com/en/products/yfiles/yed/} is a free, multi-platform program, that we used extensively to create diagrams for our report.

\paragraph{Balsamiq Mockup} - Can be found here, \url{http://balsamiq.com/products/mockups/} is an easy program to make small mockup drawings with. 

\paragraph{Intellij IDEA} - Which can be found here, \url{https://www.jetbrains.com/idea/} is a freely available Java IDE.

\paragraph{Apache Maven} - a build automation tools for java projects found here, \url{https://maven.apache.org/}. It eases the task of dependency management by allowing automatic inclusion of source packages from online repositories, and gathers project settings in a pom.xml file.

\paragraph{GanttProject} - Can be found here, \url{www.ganttproject.biz} is a free, cross-platform Gantt diagram creator.

\paragraph{Draw.IO} - Which can be found here, \url{www.draw.io} can be used to create and edit diagrams inside a web- browser.

\paragraph{Vim} - Which can be found here, \url{http://www.vim.org/} is a highly configurable text editor distributed with most UNIX systems. It can also be installed on windows.

\paragraph{paint.net} - Can be found here, \url{http://www.getpaint.net/} is a free program that expands on the windows paint program. 



\end{document}