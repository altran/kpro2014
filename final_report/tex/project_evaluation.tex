\documentclass[../document]{subfiles}

\begin{document}

\section{Project evaluation}
\subsection{Introduction}
This chapter gives an evaluation of how the project unfolded. The main focus of this evaluation will be on how the work process went regarding how we worked together as a team, the relationship with the customer and the challenges we encountered during this process. 

\subsection{Group Dynamics}
The groups for this project were assigned randomly with seven to eight group members in each group. Our group have seven group members, consisting of mixed genders and nationalities. As we have multiple nationalities within the group, English was the prefered language. Consequently, the entire project has been done in English. This section will look into how the group dynamic has been during this project.

When we first met to start planning the project, the entire group agreed to try to work together as much as possible. We also agreed that we should have a fixed meeting schedule to avoid having to plan new meeting hours every week. During the project, these meeting times sometimes changed for several reasons, but we always started the meetings at the planned place and time unless something else was decided. Because one of our group members has a full time job, he was not able to be a part of the group meetings in parts of the project. To solve this we decided to have the planned meetings without him, and have contact with him when he had the time. We also decided that he should have his own tasks to solve. 

In the beginning of the project we used some time brainstorming ideas for the project, which was a good way to start the process. We got used to working together from the start, and everyone knew the plan for the project. As the project went on to a more concrete planning phase the tasks were easily distributed between the group members because everyone had been a part of creating the idea for the project. Everyone also took responsibility to finish their own tasks, and help those who needed help. 

When we started the sprints we split the tasks within the group based on what we initially wanted to work on. Furthermore, as we wanted everyone to learn as much as possible from this project, we intended to rotate the tasks on a regular basis. However, as the project proceeded and we made changes in the idea for both the prototype and the database, most of the group members continued working on the same types of tasks. This was done because it seemed to be the most efficient choice. Consequently, the only type of task that was done by all the group members was to write documentation. Still, since most of the group members worked together, we had the chance to update each other on what had been done in the different parts of the system immediately.

As a result of this way of working, the group has handled changes in the project in a good way. Every time a change has been made the entire group has taken responsibility to finish the new tasks, and make both the prototype and the documentation as good as possible.

\subsection{The Work Process}
This section describes how the initial requirement process went, and how the subsequent scrum process went. 

\subsubsection{Requirements}
We started our work process with a requirement phase to plan the project, and document the requirements as accurately as possible before we started on the sprints. During this phase we used a lot more time than expected creating ideas and forming requirements that the customer was satisfied with. As a result the requirements phase was extended by a week compared with our project plan, which resulted in all the sprints being postponed by a working week.

\subsubsection{Scrum}
Even though we decided to use Scrum as our methodology during the preliminary study, we never actually followed the scrum methodology strictly. Before every sprint we made a sprint backlog with time estimations and priorities so that everyone would know what needed to be done during each sprint. The time estimations turned out to be a lot harder to gauge than expected. In the first sprint we underestimated the time needed to finish the tasks, and in sprint two and three we ended up overestimating the time needed in fear of underestimating again. As a result we have had more time available to work on documentation during sprint two and three than expected.

At the end of the sprint we wrote the sprint documentation and evaluated the sprint before planning the next sprint. Because we almost never did any work outside of the group meetings, we never had an actually scrum stand-up meeting to inform each other of our progress and plans forward. We all told each other what we were doing when we started working and when we changed tasks. We experienced that this way of working was efficient because we were all working together, and ended up skipping the planned weekly meetings. 

Furthermore, even though the requirements process was finished, we modified the requirements, test plan and architecture several times as the project tasks changed. The actual tasks for the next sprint was decided after the last sprint was finished and evaluated by the customer. As a result of this, it was easy to change our plans for the prototype as it became clearer to us what the customer wanted.

The main advantage we experienced related to working with scrum in this project was that the implementation phase was divided into sprints. Because we got feedback from the customer after every sprint, we also had the chance to change the path of the project when starting the next sprint.  

\subsection{Risk management}
Most of the risks predicted in the planning phase occurred one or several times during this project. This section will explain which risks that occurred and how we handled them. 

\subsubsection{Risk 2 - New Technologies}
During the course of the project this risk occurred frequently. This was predicted during risk analysis by assigning this risk a high probability. 

\paragraph{JavaFX} \ \\
From the start it was apparent that the group would have to implement a graphical interface in Java. After some initial investigation it was decided to use JavaFX, a library of which the group had little previous knowledge. According to the risk strategy, three group members gained knowledge of JavaFX by implementing a simple table view during the first sprint. The code itself and code comments served as documentation. 

\paragraph{NoSQL database} \ \\
At the customer meeting september 26, it was requested that the group replace their SQL database with a more suitable NoSQL solution. The group needed to investigate possible solutions from scratch, a demanding task that was assigned to two group members with a technical understanding. During the course of a week, they explored a considerable number of technologies involved in different solutions, documenting their usage with well-commented code snippets. The effort eventually resulted in a commented prototype implementation of the database that served as a final reference on technology usage.

\paragraph{Maven project management} \ \\
The customer also requested that the group use Maven for project management. One of the group members attended to a Skype conversation with the customer where Maven project management was addressed briefly. In addition, the two group members working on the replacement database solution integrated the project in Maven on a separate code branch. This Maven setup of this code branch was later integrated into the rest of the project. The POM files (Project Object Model) served as documentation for Maven project management. Some other group members were briefed in Maven, enough to understand the POM files. It was not deemed necessary that all group members know Maven.

\paragraph{Latex} \ \\
The group decided early to write their report in Latex. One group member with considerable Latex experience wrote a brief introduction to Latex, with links to further reading.

\paragraph{The vdvil project} \ \\
According to customer requests, the group needed to investigate the vdvil project, a previous project the customer had been involved in. One group member attended to a Skype conversation with the customer, and wrote a short note on how to set up the project for the other group members.

% Technology risks that occurred that were not mentioned:
% Git, GitHub. IntelliJ IDEA.
% Raspberry Pi central hub and sensors. Central hub Network setup

\subsubsection{Risk 3 - Use external services}
To keep track of our documentation and code we used a combination of Google Drive and GitHub. Even though these services never went down, we handled this risk by having a back up on a personal machine run by one of the group members. 

\subsubsection{Risk 4 - Sickness}
Throughout the project several of the group members got sick on one point or another. As a result of this, we sometimes lacked group members on the different meetings. We handled this by redistributing the tasks needed to be done between the remaining group members before we started working. The person that was sick was responsible to inform the rest of the group as soon as possible. 

\subsubsection{Risk 5 - Absence because of other schedules}
As all the group members have a full schedule, it was sometimes necessary for most of the group members to do something else than work on this project during the hours that we had scheduled to use on the project. To prevent this to have too much impact on the project the person missing was responsible to inform the rest of the group a soon as possible, and the tasks that were suppose to be done by the missing person were reassigned to other group members, if necessary. As one of our group members could not meet on most of the group meetings because of work, we assigned him with his own tasks and tried to communicate as much as possible with him to avoid misunderstandings. 

\subsubsection{Risk 7 - Miscommunication}
During the project we had some miscommunication both between the group and the customer, and between group members. The miscommunication often took some time to discover, which sometimes led to unnecessary work being done. However, when a miscommunication was discovered we handled it by discussing within the group why the miscommunication occurred, what it was about, and what we wanted to do next. Then we would present the plan further with the other party involved in the miscommunication. We would also ask relevant questions to the other party to get answers about the parts that were misunderstood, if there was any misunderstanding. 

\subsection{Time Estimation}

We estimated that the whole project would take 2150 hours to complete, in the end we calculated that we had used {\color{red} ?? (Most likely 1950)} hours on this project. A detailed overview of the time can be found in this subsection, both in textual format and as a table in \tableref{tab:time_estimate}

The planning of this project changed a few times throughout the duration of the project. Therefore the estimated hours are different than the hours we actually spent on the given tasks. The self-learning in the start took much more time than we expected, as we needed to connect as a group and attend lectures, in addition to brainstorming ideas for the prototype. However, during the rest of the project we spent around fifteen hours on self-study, which was what we estimated. The hours spent on planning the project in the beginning took somewhat time than we had estimated. 

Writing the requirements specification had some factors that changed the work load. At first we spent more hours than estimated on the requirements because the project needed many more requirements than we first thought. However, due to big changes in our system, we needed to discuss the changes instead of writing requirements. This led to that we worked less than half of what was estimated the third week. The fourth week we worked on the requirements for sixty hours. These hours were not included in the estimated work hours, but they were needed, to rewrite the requirements for the smart room into requirements for the visualizer.

We had planned to start sprint 1 in the fourth week, so the time estimated for the first week of sprint 1 was entirely wrong. During the next three sprints we worked less on the documentation and the implementation than estimated. This is because there were many changes done by the customer during the sprints. Some of the time we had planned to work on the sprints were used to discuss and to figure out how to solve the changes the customer wanted. We also needed to learn new frameworks for the database twice, which took time. Another factor that made the time we estimated wrong, was the nonpresent group member who did not work when we worked and only focused on the Raspberry Pi and the sensors. In the latter weeks, we added a sprint we did not plan to have. While a few members wrote the documentation as we had planned, the rest worked on the last changes in the visualizer. Therefore we spent less time on the finalization of the report than we had estimated. 

We did not estimate the time all the meetings during this project took, which was usually one hour of meeting followed by one or more hours of discussion. Overall we had many changes during the project, so we did not follow the estimated hours quite as planned. The overview of estimated hours, per each activity and week of work can be seen in a \tableref{tab:time_estimate}


\begin{table}[H]
\caption{Description}
\centering
\label{tab:time_estimate}
\begin{tabularx}{\textwidth}{|l|X|X|L{1.6cm}|X|}
	\hline
	\textbf{Task}
	&\textbf{From week}
	&\textbf{To week}
	&\textbf{Estimated hours}
	&\textbf{Actual hours}
	\\ \hline \textbf{Misc}
	&
	&
	&
	&
	\\ \hline Self study
	&1
	&13
	&195
	&282.5
	\\ \hline Planning
	&1
	&2
	&165
	&116.5
	\\ \hline Initial requirements specification
	&2
	&4
	&190
	&157.5
	\\ \hline \textbf{Sprint 1}
	&
	&
	&
	&
	\\ \hline Sprint 1 implementation
	&4
	&6
	&117
	&157
	\\ \hline Sprint 1 documentation
	&4
	&6
	&120
	&46
	\\ \hline \textbf{Sprint 2}
	&
	&
	&
	&
	\\ \hline Sprint 2 implementation
	&7
	&8
	&187
	&116
	\\ \hline Sprint 2 documentation
	&7
	&8
	&120
	&41
	\\ \hline \textbf{Pre-delivery for the examiner}
	&
	&
	&
	&
	\\ \hline Pre-Delivery for the examiner
	&7
	&8
	&30
	&21.5
	\\ \hline \textbf{Sprint 3}
	&
	&
	&
	&
	\\ \hline Sprint 3 implementation
	&9
	&10
	&124
	&77.5
	\\ \hline Sprint 3 documentation
	&9
	&10
	&120
	&130.5
	\\ \hline \textbf{Sprint 4}
	&
	&
	&
	&
	\\ \hline Sprint 4 implementation
	&11
	&13
	&50
	&19
	\\ \hline Sprint 4 documentation
	&11
	&13
	&??
	&??
	\\ \hline \textbf{Presentation and documentation}
	&
	&
	&
	&
	\\ \hline Review and finalize project
	&11
	&13
	&??
	&??
	\\ \hline Prepare Presentation
	&13
	&13
	&??
	&??
	\\ \hline \textbf{Meetings}
	&
	&
	&
	&
	\\ \hline Group meeting
	&1
	&13
	&182
	&258
	\\ \hline Advisor meeting
	&1
	&13
	&78
	&78
	\\ \hline Customer meeting
	&1
	&13
	&78
	&63
	\\ \hline 
\end{tabularx}
\end{table}

\subsection{Quality Assurance}
In the planning phase of this project we defined some quality metrics that measure how well our prototype is implemented, and how we plan to ensure that these quality metrics are met in our prototype. This section will describe which parts of the plan worked and which that did not, along with what we ended up doing instead of the plan.

To ensure high quality of the documentation and the code we planned on using online shared repositories. This was done through the entire project and it ensured that everyone had an overview of the documentation and the code at all times. Furthermore, to ensure that the code is easy to understand, even for those not familiar with the code, we planned on using already defined coding standards for Java. This included writing JavaDocs for all the classes and additional comments where that was needed. After the comments were written, the entire group could more easily get an overview of what the other group members had done. To make working with GitHub easier we planned on following some guidelines described in the quality assurance section.

To ensure high quality in our communication with the customer, and in our work in general, we planned on having weekly meetings with both the customer and the advisor. While the advisor meetings have been held on a regular basis, the customer meetings has been more sporadic. The reason is that we never found a fixed time to have the customer meetings, and ended up having to schedule meetings at different hours every week. As a result of this, some misunderstandings occurred and some meetings were canceled, leading to irregular contact with the customer.

\subsection{Customer Relationship}
During this project we have had many miscommunications with the customer. We have more than once misunderstood our task. At first we thought we were going to do something with the sensors, like creating a smart room. However, this was more than the customer asked for, as he only wanted a visualiser for the data. Therefore, we decided to tune down the project. After we tuned the project down, we started asking more concrete questions in the meetings to avoid misinterpreting what the customer actually wanted. Many changes from the customer may have been stressful, however, it has given us more documentation, and a good understanding of how a real customer can act. 

The customer was located in Oslo, therefore all the meetings we had with him were over Skype. The only other way we interacted with the customer was by mail, and we had maximum one meeting per week. Therefore, when we started to work on a sprint it often took a week before we could get feedback. This lead to us having to change the visualiser or the database more than once. 











\end{document}