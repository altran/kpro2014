\documentclass[../document.tex]{subfiles}

\begin{document}

\section{Project Reflection}

\subsection{Introduction}
This chapter contains the reflections about how we think the project unfolded. The subjects we will reflect on is the internal process, the customer and project task, the advisor, and how the course could be made better. All of these subjects are also a part of the project evaluation, but while the project evaluation contains descriptions about how the different processes went, this chapter will be about what we would have done differently and what we have learned. 

\subsection{The Internal Process}
The collaboration within the group has been good throughout the project, and we have therefore been able to make both a prototype and report that we are satisfied with. We have learned that working together, at the same time and in the same place, is a good idea. This is because we always had the opportunity to help each other when that was necessary. This made it easier to write documentation and to solve implementation issues. However, looking back at the project, there are several things we could have done differently. 

To make sure that everyone learned as much as possible about the project, we initially wanted to rotate the different tasks so that everyone would eventually work on all the parts of the system. As a result of us not doing this, we had to divide the documentation so that everyone could write about the parts they knew best. This was not a big problem, but it could have been avoided to some extent if everyone had the same understanding about the entire project. To make sure that everyone knew exactly what was going on in all the different parts of the system, we should have had regular meetings with the entire group. Because one of our group members had a full time job, he was not able to come to the group meetings. We had contact with him through another group member, but this way of working did not give the group a good overview of what he had done, and what he was doing next. Also, he did not have a complete overview over what we were doing, or what decisions we had made. This led to both us and him working based on assumptions rather than facts, which again led to some misunderstandings. One way that these misunderstandings could have been avoided is by having meetings with the entire group, or at least several members of the group. In this way, the entire group could have the exact same information about the project. Another way to avoid this could have been that the entire group wrote an agenda for the meeting between the two group members, so that nothing was forgotten. Also, the group members in the meeting should write a summary of the meeting that everyone could read. 

\subsection{The Customer and the Project Task}
Even though we had some miscommunications with the customer, the relation we had with him was generally good throughout the entire project. The miscommunications taught us a lot about how to communicate with a customer, which will come in handy later in life. One of the most important lessons we learned was that we should have had a fixed meeting schedule with the customer. With a fixed meeting schedule we would have been able to avoid some misunderstandings about the meeting time, and we would therefore been able to have meetings more frequently than what we had during this project. With more meetings we could have avoided having to make changes to parts of the prototype that we had worked a lot on because it was not quite right. Another lesson was that we should be more specific from the start when talking to the customer. No information about the project should be unsaid so that there are no misunderstandings about what we plan to do and what we have done.  

During this project, one of the suggestions made by the customer, and one of the biggest lessons we have learned about implementation, is to write the code as modifiable as possible. All the changes that have been made during the project would not have been as easy to carry out without modifiable code. Because of this we now know, after experience, that one should always try to make implementations as modifiable as possible in case of any future changes. 

\subsection{The Advisor}
The advisor has been a good resource for this project. He has both been giving us valuable feedback about the documentation, and advice about how we should communicate with the customer. The meetings we have had with him also taught us how valuable it is to have a regular meeting schedule, so that there are no misunderstandings, as well as an even flow of feedback. Furthermore, we have learned of the importance of a manager-like figure in a project like this.

\subsection{Suggestions for Improvement}
There are many good aspects of how this course is carried out. Each of the groups has the freedom to decide how to do the project. The advisor is there to guide the groups, which is helpful for knowing whether or not the project is on the right track. However, some aspects of the course could benefit from some change. The mandatory team building session came too late in the project. We had already been working together for a while and have gotten to know each other before the session. The session should be done with only a couple of groups at once. With a smaller group, we could have had a better discussion not only in the group, but also between the groups. Finally, the guest lecture about technical writing in english could benefit from being one of the first guest lectures. The lecture was very well written and presented, and was quite helpful, but not many students showed up. Often, many students only attend the first few lectures, and then prioritize other work.
\subsection{Summary}
Over the duration of this project we have all gained a lot of new experiences. The group has been cooperating well, even if not everyone has been able to attend the group meetings. We have been able to collaborate on several tasks, help each other when that has been necessary, and handle challenges in a reflected manner. When we planned the project and the initial requirements, we pictured the development in the project in a quite different way than what actually happened. But because of our detailed risk management plan and initial project plan, dealing with any of the changes that occurred during the project went quite easy. All in all we are very pleased with what we have been able to achieve in this project.

\end{document}
