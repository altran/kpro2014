\documentclass[../document]{subfiles}

\begin{document}

\section{Test Plan}
\label{test_plan}
\subsection{Unit Testing}
We are planning to use JUnit to do the unit testing. Every single test is done by the developer, which means that the tests for a module are written by the developer who made the module. 

\begin{table}[H]
\caption{Unit Test Plan}
\centering
\begin{tabularx}{\textwidth}{|l|X|L{2.4cm}|X|}
	\hline
	Modules
	&Description
	&Input
	&Output
	\\ \hline Visualiser
	&The visualiser will be able to showcase the data which is pulled from the database.
	&When the program starts, it will obtain data from the database.
	&Visualisation is displayed in the correct fashion with the correct data.
	\\ \hline Central Hub
	&The central hub collects data from the sensors using a \gls{DASH7} protocol and uses \gls{REST} \gls{API} to store it in the database.
	&Data from sensors.
	&Data into the database
	\\ \hline Database
	&The database contains all the information gathered from the central hub.
	&Data from the central hub.
	&Data can be pulled by the visualiser from the database.
	\\ \hline Sensor
	&The sensors create and send data based on the environment they are in.
	&Environment observation.
	&Data about their environment.
	\\ \hline 
\end{tabularx}
\end{table}

\subsection{Test coverage}
Each unit testing should have atleast 60 percent code coverage and full state coverage. The reason we are only doing 60 percent is because our lack of time and the lack of testing experience at professional level, also we belive the code is rather tolerant of unused codes and so on. We do belive that a full state coverage is needed and this is because we want to animate all the different state of sensor movement, be it right or wrong. 

\subsection{Integration Testing}
In this section we will collect the different modules of our prototype together and test if they are able to correctly work. Do note that requirement 3.1 and 3.2 is needed for everything to work, but it is expected for it to work since it comes from the customer. Therefore we are not gonna mention it directly in the coming up text.

\paragraph{Modules that need to be integrated}
\begin{itemize}
	\item
	Visualiser from database
	\item
	Central hub to database
	\item
	Central hub from sensors
\end{itemize}

\paragraph{Tests that need to be done at each integration}
\begin{itemize}
	\item
	Visualiser from database:
	\begin{itemize}
		\item
		Visualiser needs the right call to get information from database.
		\item
		Visualiser can get the data from the database.
	\end{itemize}
	\item
	Central hub to database
	\begin{itemize}
		\item
		Central hub will connect to database
		\item
		Central hub will post the new information into the database
	\end{itemize}
	\item
	Sensor to central hub
	\begin{itemize}
		\item
		Central hub will pull the data from sensors
		\item
		Central hub will listen to the sensors
	\end{itemize}
\end{itemize}

\begin{table}[H]
\caption{Integration Test Plan}
\centering
\begin{tabularx}{\textwidth}{|L{2cm}|l|L{2.2cm}|X|}
	\hline
	Test name
	&Test ID
	&Requirement(s) ID
	&Description
	\\ \hline Hardware
	&T0
	&None
	&The hardware is provided by the customer and it should not be our concern, nor do we have enough knowledge about it, to test it.
	\\ \hline Central hub gathering and processing
	&T1
	&2.1, 2.3
	&This test will gather information from the hardware and then process it into a set of data which is sorted by sensorID.
	\\ \hline Central hub data storage
	&T2
	&1.1, 2.2
	&After processing the data, it will be pushed forward into the database and stored. This process will be using \gls{REST} \gls{API}.
	\\ \hline Central hub hotjoin
	&T3
	&2.4
	&The central hub can take in new sensors or remove sensors from it at runtime.
	\\ \hline Visualiser taking data from database
	&T4
	&1.2, 4.1
	&The visualiser will pull known data from the database and we will check if the values are correct.
	\\ \hline Visualiser displaying the data
	&T5
	&4.2.1, 4.2.2, 4.2.3
	&Visualiser displaying the correct data that it pulled from the database. The data should be displayed correctly both for the table, and for the animation.
	\\ \hline Visualiser switches what data to visualise
	&T6
	&4.3
	&The user should be able to manipulate what data to visualise at any time by using checkboxes.
	\\ \hline 
\end{tabularx}
\end{table}

\subsection{System Test}

\begin{table}[H]
\caption{System Test Scenario 1}
\centering
\begin{tabularx}{\textwidth}{|l|X|}
	\hline
	Test Name
	&Test Scenario 3
	\\ \hline Description
	&First the sensors, central hub and the database need to be set up. We will see if the sensors gather the correct data, and if they do, how the central hub pulls the data from the sensors, processes the data and then stores it.
	\\ \hline Enviroment
	&Run time
	\\ \hline Test case involved
	&T0, T1, T2
	\\ \hline Dependency
	&None
	\\ \hline Input
	&The data of surrounding environment.
	\\ \hline Output
	&The correct data is stored into the database
	\\ \hline Acceptance
	&If the data stored is correct and in the format we wanted.		
	\\ \hline 
\end{tabularx}
\end{table}

\begin{table}[H]
\caption{System Test Scenario 2}
\centering
\begin{tabularx}{\textwidth}{|l|X|}
	\hline
	Test Name
	&Test Scenario 2
	\\ \hline Description
	&A new sensor is added during runtime and we see if the data can be registered by the central hub. After that remove the sensor and see if this affects rest of the system. 
	\\ \hline Enviroment
	&Run time
	\\ \hline Test case involved
	&T3
	\\ \hline Dependency
	&Test scenario 1
	\\ \hline Input
	&A new set of sensor data / removing a sensor
	\\ \hline Output
	&If a new set of sensor data is added, it will be shown independently on the next pull from the central hub. If the sensor is removed the data in the database will reflect that by not updating anything.
	\\ \hline Acceptance
	&When the new sensor set is updated correctly with correct data. And if we remove any sensors, the database will not update anything random and the whole system should not crash.
	\\ \hline 
\end{tabularx}
\end{table}

\begin{table}[H]
\caption{System Test Scenario 3}
\centering
\begin{tabularx}{\textwidth}{|l|X|}
	\hline
	Test Name
	&Test Scenario 3
	\\ \hline Description
	&This test will cover all functions for the visualiser. The visualiser will pull data from the database and then show them on the GUI. This will be done both in the traditional table format and an animation format. The user should also be able to chose what to show by interacting with the program.
	\\ \hline Enviroment
	&Run time
	\\ \hline Test case involved
	&T4, T5, T6
	\\ \hline Dependency
	&Test scenario 2
	\\ \hline Input
	&The data of the surrounding environment.
	\\ \hline Output
	&The correct data is shown on the visualiser. And the user interaction works (meaning one can turn off what data one does not want to see).
	\\ \hline Acceptance
	&The data is correctly shown both on the table and in the animation. The animation needs to be smooth (meaning at least 24 fps). The user interaction will not have any impact other than hiding or showing a certain data type.
	\\ \hline 
\end{tabularx}
\end{table}

\begin{table}[H]
\caption{System Test Scentario 4}
\centering
\begin{tabularx}{\textwidth}{|l|X|}
	\hline
	Test Name
	&Test Scentario 4
	\\ \hline Description
	&This test will be an overall test that will last over a long period of time. We will set up the system in a closed environment and observe the changes of data over a long period of time (twelve or more hours, depending on the battery life). During this time the data will be stored and shown.
	\\ \hline Enviroment
	&Run time
	\\ \hline Test case involved
	&T0,T1,T2,T3, T4, T5, T6
	\\ \hline Dependency
	&Test scenario 3
	\\ \hline Input
	&The data of the surrounding environment.
	\\ \hline Output
	&The correct data is shown on the visualiser.
	\\ \hline Acceptance
	&Same as in test scenario 3, just that the system has to be able to run over the whole period of time without any failure that shuts down the system, in other words failure handling is a concern in this case.
	\\ \hline 
\end{tabularx}
\end{table}

\subsection{Short System Test Table}
\begin{table}[H]
\caption{Short System Test Table}
\centering
\begin{tabularx}{\textwidth}{|l|X|X|}
	\hline
	Scenario nr
	&Acceptance
	&Result
	\\ \hline 1
	&The data stored is correct and in the format we wanted.
	&The sensor data coming in is in raw data format and then was saved as json fomat.
	\\ \hline 2
	&When the new sensor set is updated correctly with correct data. And if we remove any sensors, the database will not update anything random and the whole system should not crash.
	&The new sensor is working and treated by the system at the same level as any other sensors, the system and the database is also running without any problem.
	\\ \hline 3
	&The data is correctly shown both on the table and in the animation. The animation needs to be smooth (meaning at least 24 fps). The user interaction will not have any impact other than hiding or showing a certain data type.
	&We cross check the data with table and what we see in the animation and they do match. The fps was stable 60 with drop down to 55 or so sometimes, we do belive thats a problem with the hardware it self. When user can only hide or show certain data type any other interraction is not possible.
	\\ \hline 4
	&Same as in test scenario 3, just that the system has to be able to run over the whole period of time without any failure that shuts down the system, in other words failure handling is a concern in this case.
	&The long period of time went well, we did not ecounter any problem and the memory use was stable at 150MB all the way, thanks to java garbage collector.
	\\ \hline 
\end{tabularx}
\end{table}

\subsection{User test}
We do not have a large user base to perform user tests due to our lack of time and resources. Therefore most of our user testing was done inside the team, with customer and adviser as the only two outsiders who have seen our implementation so far. Although this is not enough to say if our UI is up to par, but it did give us a lot of help during the developement when something is wrong.

\subsection{Performance test}
Performance is one of the few non-functional goal we have, and the only one we are able to test fully. We are planning to have a 24 hour test from test scenario 4, which is targede directly at performance. From that test scenario we should get a good grasp on how good the performance is of our system.


\end{document}
