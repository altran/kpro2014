\documentclass[../document.tex]{subfiles}

\begin{document}
\section{Project Directive}

\subsection{Project Name}
The project given to us by the customer is labeled “Internet of Things”. “Internet of Things” is a conceptual idea that involves using cheap sensors to measure different data within an environment and create a showroom. We will use the showroom to interactively present, or even get involved with, the data presented by the manipulation of the sensors.

\subsection{Project Sponsor}
The customer for this project is Altran Norge AS. They are a part of an international group that offers innovative and high-tech engineering consulting. According to their website, their mission is to help organizations and companies creating and developing new products and services. They are well established in Enterprise Content Management, and embedded software and electronics design. However, they also offer other services in IT and technology. Altran Norge AS is responsible to provide us with the hardware necessary to complete the project.

\subsection{Project Purpose}
The goal of the project is to use and test the sensors that our sponsor, Altran AS. should provide. In the case Altran AS. is unable to provide the sensors, we will need to emulate them somehow. The sensors are able to detect and measure temperature, humidity, lighting, pressure and more. We are to present and visualize this data in an interesting fashion. More concretely, we want to showcase the possibilities of the sensors in some visually appealing way. We are to create a network of such devices that can communicate within any given environment within itself and a central hub.

\subsection{Involved Parties}
Here we detail the parties involved within this project, in what way they are involved, their contact information, and their position within the project.


\subsubsection{The Customer}
\begin{tabular}{ll}
\hline
Name					&	Stig Lau\\
Company				&	Altran Norge AS\\
Position				&	Consultant\\
Contact Information		&	Stig.Lau@altran.no\\
\hline
\end{tabular}

\subsubsection{Students}
\begin{tabular}{ll}
\hline
Name				&	Contact Details\\ \hline
Besleaga, Catalin		&	catalinb@stud.ntnu.no\\
Ludvigsen, Niklas		&	niklasl@stud.ntnu.no\\
Kirkhus, Julie Johnsen	&	juliejk@stud.ntnu.no\\
Remøy, Maria Aune		&	mariaare@stud.ntnu.no\\
Sun, Shimin			&	shimins@stud.ntnu.no\\
Sutterud, Audun		&	audunsu@stud.ntnu.no\\
Zikic, Nenad			&	zikic@stud.ntnu.no\\
\hline
\end{tabular}

\subsubsection{Group Adviser}
\begin{tabular}{ll}
\hline
Name				&	Mohsen Anvaari\\
Position			&	Adviser, PhD-Student\\
Contact Information	&	mohsena@idi.ntnu.no\\
\hline
\end{tabular}

\subsection{Project Background}
The “Internet of Things” is a project that is in its early stages. The project is anticipated to change the world by implementing smarter systems. These systems are sensors that will detect different changes within an environment, and communicate this data to a central hub. This hub will analyze the data and allow us to manipulate the environment around us. The key of the project is what can and what can not be done with the sensors. Which systems we can improve or replace and where we can create new opportunities. It is also necessary for us to be able to showcase this data visually and, depending on the system, show how the system works in real time.

Our main focus of the project will be to visualize the data gathered by the sensors. We want to be able to showcase the data we gather on a screen through image manipulation. The data from the sensors such as light or temperature will directly influence and manipulate this image in real-time. The visualization must be interesting and must directly showcase how the multiple sensors work together to gather information from the environment.

\subsection{Project Goals}
The overarching goal of this project is to think of ideas that can visualize the data gathered by the sensors in an interesting fashion. Another goal is to make sure the infrastructure of the system is highly modular. In this way the system may be re-used in any sort of environment. It is also necessary to note where this system can not be used and where the current systems are already a better solution than our own system.

The specific goal of our team is to design a showcase, or visual representation of data. This data is gathered either in real-time or through the course of a day and shown in a time lapse. How the data is gathered is depending on what type of data we would like to showcase. 

\subsection{General Terms}
The main limitation of this project is the lack of hardware. The sensors are currently in production and their availability will limit our work within the implementation and testing phase. Time is another limitation. We have to manage our working time well in order to finish this project in a satisfying condition.

\subsection{Planned effort}
The project started August 26th, however, the project was introduced on the August 28th by our customer, and will end with a final presentation November 20. The project lasts for exactly twelve weeks, and each group member is expected to work approximately 25 hours every week during this period. Thus, the total number of hours per team member will be approximately 300 hours. Because the grout consists of 7 members the total size of the project will be 2100 person-hours. 

\subsection{Schedule of Results}
The requirements should be done by September 14, so that the first sprint can start at September 15. The project will have three sprints, each of which lasts for two weeks. This means that the first sprint should be done by September 28, the second sprint should be done by October 12 and the third sprint should be done by October 26. The final presentation is held, and the report is handed in at November 20.

\subsection{Similar Projects}
With the idea being fairly abstract we have to consider many variables and fields of study in order to fully investigate all the possibilities of this project. There exist companies that are involved in the business of “Internet of things” and we will research those further in this documentation. Because our goal is to present the data in a visual way, and not actually use the data in any other form, projects like this one may often be internal and similarities may be hard to find. This idea will be more concretely addressed in the part of the project dealing with preliminary studies.

\end{document}