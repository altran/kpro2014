\documentclass[../document.tex]{subfiles}

\begin{document}

\section{User Installation Guide}

\subsection{System setup}
Setting up the system involves setting up different individual components. These components are the central hub, the client and the visualiser. In addition, a network needs to be set up for communication between the central hub, the client and the visualiser.

\subsection{Network setup}
As the central hub can only run on a Raspberry Pi, a network is needed to connect it to the rest of the system. The Raspberry Pi is connected to the network using a cable. The Raspberry Pi is configured to get an IP address automatically through DHCP, so the network needs to have a DHCP server available. Normally the DHCP server is a router, but a laptop running Linux can also be setup to function as one. Regardless, it is recommended to setup your own dedicated network using either a router or a laptop, as it eases the next step considerably. Instructions for setting up a laptop running Linux as a DHCP server can be found here:
	\url{http://www.cyberciti.biz/faq/howto-ubuntu-debian-squeeze-dhcp-server-setup-tutorial/}
Setting up a DHCP server on windows involves a considerable amount of work, and is not recommended.

After the network is set up with a DHCP server, the different systems running the components will be assigned IP addresses automatically when they are connected.

{\color{red} <picture here>}

\subsection{Acquiring IP addresses}
In order to start the components of the system, we need to know the IP addresses of the central hub and the client. As they are assigned dynamically by the DHCP server, it is not possible to known them in advance.

For the central hub, the simplest way to discover the address is by trial and error. For this we need to know the address pool used by the DHCP server, so take note of this when setting up the network. The idea is to test subsequent addresses from this pool until the address of the central hub is found. The central hub runs a REST-based server, and the easiest way of testing an address is to type in the address in a browser. If the address is correct, this will open the configuration service of the central hub. It is likely that the central hub was leased an address in the lower end of the range, so it is a good idea to start there. For example, if the setup is performed on a local network, a typical lower end of the range address is 192.168.1.1.

If a laptop is used as a DHCP server, another option is also available. This involves viewing the log messages from the running DHCP server as they appear. The DHCP server will write the addresses it leases to the log, allowing you to read the address of the central hub.

It is also necessary to get the IP address of the client server. Assuming the client is run on a computer with easy access to a terminal, this can be found using the command-line utility ipconfig in Windows, or ifconfig in OS X or Linux.
{\color{red} <add example of terminal command here?>}


\subsection{Starting the system components}
At this point all devices are powered on, they are connected to the network, and the IP address of the central hub and the client server has been acquired. The final step is to start the individual system components. These include the visualiser, the client server, and a small script running on the Raspberry Pi. The central hub is started automatically by the Raspberry Pi on boot. It is important that the client server already runs before the visualiser is started. Apart from that, the components can be started in any order. However, it is recommended to start them in the order specified below.

\subsection{Starting the client server}
The client server receives sensor samples from the central hub, storing them in a database. In order to start the client server, open a terminal and navigate to the folder where the file the executable .jar file is located. Next, execute the .jar using the command
\begin{lstlisting}
$ java -jar IoT-service-0.9.1-SNAPSHOT-with-deps.jar
\end{lstlisting} The client should now start printing log messages.

\subsection{Starting the script}
The script running on the Raspberry Pi along with the central hub is used pull sensor samples from the central hub and push them to the client server. It is a Bash script that uses the command-line tool cURL to communicate with the central hub and the client. This script resides on the central hub under the path /home/root/push.sh. The recommended way to start the script, is to log on to the Raspberry Pi through ssh, using ‘root’ as both username and password. After logging in, the script is started with the command
\begin{lstlisting}
$ ./pull.sh <central hub ip> <client server ip>
\end{lstlisting}
where the IP of the central hub and the database is specified as arguments.

The script can also be run on any computer connected to the network. Using a bash shell it can be pulled from the central hub with the command
\begin{lstlisting}
$ scp root@<central hub ip>:/home/root/pull.sh .
\end{lstlisting}
and then executed as mentioned above.

\subsection{Starting the visualiser}
{\color{red} <Todo: Write about how the visualiser is started. Explicitly mention where you enter the IP address of the client server>}


\subsection{System configuration}
Some aspects of the different system components can be configured. In particular, the central hub can be configured in how it pulls data from the sensors. This configuration interface is accessed with a web browser, typing the IP address of the central hub into the address bar. Visiting the IP address should redirect the browser to a login screen. As before, use as login credentials ‘root’ as both username and password.

\end{document}
